
\section{Introduction}

In this report, we delve into the realm of Dynamic Programming and Reinforcement Learning (DPRL) applied to the {} problem. Our specific focus lies on the implementation of the DPRL Method and %. We will apply these two evolutionary strategies to train a generalist agent and evaluate their performance based on the evolutionary progress of the entire population and the adaptability of the best individuals to beat enemies.

\subsection{Experimental Hypotheses}
Our focus centers on two critical metrics: coverage speed and the success of the best solutions. In the context of the mechanisms of the two EAs, we posit the following hypotheses:

\begin{enumerate}[label=(\roman*)]
    \item\textbf{Hypothesis 1:} The Differential Evolution (DE) method will exhibit a higher coverage speed compared to the Tournament method.
    \item\textbf{Hypothesis 2:} The ultimate solutions in the population of the Tournament method will demonstrate better diversity after same generations of DE when coverage happens.
\end{enumerate}
%\subsection{Description of model}

\section{Methods}
%Methods: explain how your algorithms work and its motivation, parameter settings, experimental setup, fitness function, budget, etc. Make sure everything is reproducible with the information presented!

%In this section, we provide an overview of the essential elements of our evolutionary algorithms (EAs) and experiment.
\subsection{Algorithm Design}

%some setting
\begin{table}
    \centering
    \caption{Parameter applying to both EAs}
    \label{tab:my_label}
    \begin{tabular}{@{}cc@{}}
     \hline %\midrule
        Parameter & Value \\ \hline
        Population Size & 100 \\
        Generation & 40 \\
        Offspring (per gen) & 200 \\
        Crossover Prob & 100\% \\
        Mutation Prob & 20\%  \\
        Tournament members (k) & 2 \\
        Opponent size (Bound-Robin) & 10 \\ \hline
    \end{tabular}
\end{table}

\subsubsection{Initialization}


\subsubsection{}
%For evaluating the quality of solutions, we use a default fitness function provided by the framework. This can minimize the risk of overfitting due to the small training set of 8 enemies. Specifically, the individual fitness is calculated by subtracting the standard deviation from the mean individual fitness. The fitness function is defined as:
%some fomula
\begin{align*}
    individual\_fitness & = 0.9 * (100 - enemy\_hp)
    \\ &+ 0.1* player\_hp - log (time)
\end{align*}

\subsubsection{Tuning and Unsolved problems}
\label{sec:improve}

\subsubsection{Implementation}


\subsection{Experimental design and setup}
In our experiments, we set both evolutionary algorithms (EAs) to 40 generations and conducted 10 runs for each on two distinct groups of enemies, namely pairs <2,6> and <7,8>. This choice aligns with the enemy group used in Miras' genetic algorithm, allowing for meaningful comparisons.

Following training, we tested the best solutions obtained from each EA on all the enemy types for a total of 10 runs. Each enemy was tested 5 times to calculate the mean fitness value.
The Gain for each EA in testing is measured with 
\begin{equation}
    g = \sum \limits_{i = 1}^{n} p_i - e_i
\end{equation}
The selection of 40 generations was deliberate. In the case of the enemy group <2,6>, this choice allows us to observe a meaningful progression in mean fitness, highlighting any significant differences between the two EAs. However, in the case of the enemy group <7,8>, training progresses rapidly. Consequently, we opt to plot the results for only the first 10 generations, although all experiments involve the same maximum generation limit during training.

\section{Results and Discussion}
%figures
\begin{figure*}[htbp]
    \centering
    \begin{subfigure}[htbp]{0.41\textwidth}
        \centering
        %\includegraphics[width=\textwidth]{fig/}
        %\caption{Enemy group <7,8>}
        \label{fig:gain_1}
    \end{subfigure}
    \hfill
    \begin{subfigure}[htbp]{0.55\textwidth}
        \centering
        %\includegraphics[width=\textwidth]{fig/}
        %\caption{Enemy group <2,6>}
        \label{fig:gain_2}
    \end{subfigure}
    %\caption {Fitness of 2 EAs over multiple enemies}
    \label{fig:gain}
\end{figure*}
\subsection{Results}
\subsection{Discussion}
\section{Conclusion}
In this report, we have %presented a thorough introduction to the design and underlying motivation of the two Evolutionary Algorithms (EAs). Through a comparative analysis of their performance in two distinct experiments, we have unveiled their strengths, potential drawbacks, and provided insights into when each EA is most suitable for a given scenario.
%\newpage
